\begin{itemize}
    \item \textbf{Procesador:} AMD Ryzen 3 3200G
    \item \textbf{Memoria RAM:} 16 GB DDR4 3000 Mhz
    \item \textbf{Almacenamiento:} SSD 512 GB
    \item \textbf{Sistema Operativo:} Ubuntu 24.04.2 LTS
    \item \textbf{Compilador:} g++ con flags \texttt{-std=c++17 }
\end{itemize}
El formato de entrada para los casos de algoritmos de ordenamiento consistió en distintos arreglos almacenados en archivos \texttt{.txt}, generados mediante el código provisto por los ayudantes.

\vspace{0.5em}

Para los algoritmos de multiplicación de matrices, el formato de entrada consistió en dos archivos \texttt{.txt} que contienen las matrices correspondientes, también generados con el código entregado por los ayudantes.

\vspace{0.5em}

La ejecución de los programas se realizó utilizando un \texttt{Makefile}, compilando con el comando \texttt{make} en un entorno Linux (como WSL o consola nativa), y posteriormente ejecutando los binarios generados: \texttt{./sorting} para los algoritmos de ordenamiento, y \texttt{./matrix\_multiplication} para los algoritmos de matrices.